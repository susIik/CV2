%%%%%%%%%%%%%%%%%%%%%%%%%%%%%%%%%%%%%%%
% Deedy CV/Resume
% XeLaTeX Template
% Version 1.0 (5/5/2014)
%
% This template has been downloaded from:
% http://www.LaTeXTemplates.com
%
% Original author:
% Debarghya Das (http://www.debarghyadas.com)
% With extensive modifications by:
% Vel (vel@latextemplates.com)
%
% License:
% CC BY-NC-SA 3.0 (http://creativecommons.org/licenses/by-nc-sa/3.0/)
%
% Important notes:
% This template needs to be compiled with XeLaTeX.
%
%%%%%%%%%%%%%%%%%%%%%%%%%%%%%%%%%%%%%%

\documentclass[letterpaper]{deedy-resume} % Use US Letter paper, change to a4paper for A4

\begin{document}

%----------------------------------------------------------------------------------------
%	TITLE SECTION
%----------------------------------------------------------------------------------------

% \lastupdated % Print the Last Updated text at the top right

\namesection{Evo}{Annus}{ % Your name
% \urlstyle{same}\url{https://www.linkedin.com/in/tavo-annus-4a5631171/} \\ % Your website, LinkedIn profile or other web address
16.10.2001 | Viimsi, Harjumaa \\
\href{mailto:evo.annus@gmail.com}{evo.annus@gmail.com} | +372 5692 6727 % Your contact information
}

%----------------------------------------------------------------------------------------
%	LEFT COLUMN
%----------------------------------------------------------------------------------------

\begin{minipage}[t]{0.33\textwidth} % The left column takes up 33% of the text width of the page

%------------------------------------------------
% Education
%------------------------------------------------

\section{Haridus}

\subsection{TalTech}

\descript{Tootearendus ja Robootika bakalaureus}
\location{2020 - 2023}
Keskmine hinne: 5.0 \\
Cum Laude


\sectionspace % Some whitespace after the section

\subsection{Tallinna Reaalkool}

\descript{Reaal-programmeerimise õppesuund}
\location{2017 - 2020}
Hõbemedal

\sectionspace % Some whitespace after the section

\subsection{Viimsi Keskkool}
\location{2008 - 2017}

\sectionspace % Some whitespace after the section

%------------------------------------------------
% Links
%------------------------------------------------

\section{Lingid}

\href{https://www.linkedin.com/in/evo-annus-094362231/}{\bf LinkedIn} \\
\href{https://susiik.github.io/}{\bf Portfell} \\
\href{https://github.com/susIik}{\bf Github} \\

\sectionspace % Some whitespace after the section

%------------------------------------------------
% Skills
%------------------------------------------------

\section{Oskused}

\subsection{Keeled}

Eesti keel - Emakeel \\
Inglise keel - C1 \\
Vene keel - Suhtlustasandil \\

\sectionspace % Some whitespace after the section

%------------------------------------------------

\subsection{Juhiload}

B - kategooria \\

\sectionspace % Some whitespace after the section

\subsection{Erialased oskused}

\sectionspace

\descript{Mehaanika}
\location{CAD / CAM}
Solidworks \textbullet{} Siemens NX \textbullet{} Solid Edge\\
\location{Keevitamine}
MIG/MAG \textbullet{} Käsikaarkeevitus \\
\location{3D printimine}
\location{FEM}
\location{Hüdraulika / Pneumaatika}

\sectionspace

\descript{Elekter}
\location{Jootmine}
\location{Mikrokontrollerid}

\sectionspace

\descript{Programmeerimine}
Python \textbullet{} C \textbullet{} C++ \\
Java \textbullet{} TypeScript \\
Matlab



\sectionspace % Some whitespace after the section

%----------------------------------------------------------------------------------------

\end{minipage} % The end of the left column
\hfill
%
%----------------------------------------------------------------------------------------
%	RIGHT COLUMN
%----------------------------------------------------------------------------------------
%
\begin{minipage}[t]{0.66\textwidth} % The right column takes up 66% of the text width of the page

%------------------------------------------------
% Experience
%------------------------------------------------

\section{Töökogemus}

\runsubsection{Küberväejuhatuse IKT keskus}
\descript{| Tarkvaraarendaja}

\location{September 2023 - ...}
\vspace{\topsep} % Hacky fix for awkward extra vertical space
\begin{tightitemize}
  \item Arendan veebirakendust, mille abil saab genereerida konfiguratsioonifaile ruuteritele ja switchidele.
  \item Uute töötajate väljaõpetamine
\end{tightitemize}

\sectionspace % Some whitespace after the section

%------------------------------------------------

\runsubsection{Neptune First}
\descript{| Mehaanikainsener}

\location{Aprill 2022 - ...}
%\vspace{\topsep} % Hacky fix for awkward extra vertical space
\begin{tightitemize}
  \item Tiimiga ehitame andurriba TrimSense, mis võimaldab purjekal teada saada täpse purje kuju ja seega optimeerida purje trimmi.
  \item Projekteerin \textbf{Solidworks} tarkvara abil detaile ja valmistan need \textbf{3D printimise} teel.
  \item Optimeerin seadme \textbf{tootmisprotsessi}.
  \item Kohandan seadme disaini, et vähendada tootmiskulusid ja muuta seade vastupidavamaks.
  \item Valin ostutooteid ja \textbf{suhtlen ettevõtetega} vajalike detailide toomiseks.
\end{tightitemize}

\sectionspace % Some whitespace after the section

%------------------------------------------------

\runsubsection{Milrem Robotics}
\descript{| Mehaanikainseneri praktika}

\location{Juuli 2022}
%\vspace{\topsep} % Hacky fix for awkward extra vertical space
\begin{tightitemize}
  \item Projekteerisin \textbf{Solidworks} tarkvara abil THeMIS platvormile ühilduvat Tethered Follow-Me juhtimissüsteemi.
  \item Prototüübi jaoks vajalikud detailid valmistasin \textbf{3D printimise} teel.
  \item Valisin vajalikud ostutooted, et vähendada eridetailide valmistamise vajadust.
  \item Monteerisin kokku lõpliku toote ja paigaldasin selle THeMISele.
  \item \textbf{Testisin} koos teiste projekti tiimi liikmetega valminud prototüüpi ja muutsin disaini vastavalt vajadusele.
\end{tightitemize}

\sectionspace % Some whitespace after the section

%------------------------------------------------

\runsubsection{Kitman Thulema}
\descript{| Mehaanikainseneri praktika}

\location{Juuni 2022}
%\vspace{\topsep} % Hacky fix for awkward extra vertical space
\begin{tightitemize}
  \item Disainerite jooniste alusel koostasin \textbf{Solid Edge} tarkvara abil tootmisesse minevate \textbf{lehtmetallist} ja \textbf{puidust} toodete mudelid ja joonised.
  \item Valisin \textbf{materjale} ja \textbf{tootmisprotsesse} lähtuvalt kliendi nõuetest tootele.
  \item Vastutasin \textbf{3D printeri} töökorras olemise ja sellega detailide printimise eest.
\end{tightitemize}

\sectionspace % Some whitespace after the section

%------------------------------------------------

%------------------------------------------------
% Projects
%------------------------------------------------

%\section{Projektid}

%\runsubsection{Ratta 3D mudel}
%\descript{| Kooli projekt}
%
%\location{2020}
%\vspace{\topsep} % Hacky fix for awkward extra vertical space
%\begin{tightitemize}
%  \item Grupiprojekti raames disainisime ja modelleerisime \textbf{Solidworks CAD} tarkvara abil jalgratta.
%  \item Mina modelleerisin käiguvaheti, piduri, sadula ja keti. Lisaks modelleerisin ka mõningaid väiksemaid detaile.
%  \item Antud aastal oli meie projekt paremuselt teisel kohal.
%\end{tightitemize}
%
%\sectionspace % Some whitespace after the section

%------------------------------------------------


%\runsubsection{Elektrirula}
%\descript{| Isiklik projekt}
%
%\location{2021 - 2023}
%\begin{tightitemize}
%\item Alustasin projektiga, sest tahtsin luua elektrirula, millega saab sõita ilma kiiruse juhtimispulti käes hoidmata.
%\item Kiiruse regulleerimise jaoks saadakse andmed \textbf{tensotajuritest} (strain gauge), mis on paigutatud rula väändtelgedele.
%\item Mootori kiiruse reguleerimiseks kasutasin \textbf{Arduinot}, mis saab andmed sensoritelt ja väljastab vajaliku PWM signaali.
%\end{tightitemize}
%
%\sectionspace % Some whitespace after the section

%------------------------------------------------

%\runsubsection{TalTech Student Satellite}
%\descript{| Mehaanikainsener}
%
%\location{2022 - 2023}
%\begin{tightitemize}
%\item Tiimiga ehitasime \textbf{PocketQube} tüüpi satelliiti, mille eesmärgiks oli kosmoses testida uut tüüpi päikesepaneeli ja koguda kuutolmu.
%\item Disainisin \textbf{Solidworks} tarkvara abil satelliidile tiibu, millel paiknevad päikesepaneelid.
%\item Aitasin kaasa teiste mehaanika- ja tootearendusalaste küsimuste lahendamisel.
%\end{tightitemize}

%------------------------------------------------

%\runsubsection{Autonoomne paat}
%\descript{| Robotiklubi projekt}
%
%\location{2022}
%\begin{tightitemize}
%\item Grupitööna disainisime ja ehitasime paadi, mis peab läbima etteantud rada võimalikult kiiresti.
%\item Paadi kere ja kõikide osade ühenduslülid on modelleeritud \textbf{Solidworksis} ja \textbf{3D prinditud}.
%\item Paadi elektroonika süsteemide kontrollimiseks kasutasime \textbf{STM32 nucleo f303k8}, mis on programmeeritud \textbf{C keeles}.
%\item Sensoritena on kasutusel \textbf{IR sensorid}, millega saab mõõta kaugust mingist objektist ja selle abil arvutada paadi optimaalse sõiduteekonna.
%\end{tightitemize}

% \begin{tightitemize}
%   \item \textbf{\href{https://github.com/kilpkonn/neowatch}{neowatch}} - Creator of a Rust alternative to \textit{watch} command with addons.
%   \item \textbf{\href{https://github.com/DEVELOPEST}{gtm}} - One of the creators of an open source time tracking app \textit(similar to wakatime).
%   \item \textbf{\href{https://github.com/kilpkonn/SportsApp}{SportsApp}} - Creator of an open source Android app for tracking outdoor trainings.
%   \item \textbf{\href{https://github.com/kilpkonn/LonePlayer}{LonePlayer}} - One of two creators of 2D platformer game.
%   \item \textbf{\href{https://github.com/kilpkonn/GomokuGUI}{GomokuGUI}} - Creator of an open source Gomoku game made for AI \textit{(TalTech uses it for Java course assingment)}.
%   \item \textbf{\href{https://veloren.net/}{veloren}} - Contributor, improved camera clipping, misc other minor improvements.
% \end{tightitemize}

%\sectionspace % Some whitespace after the section

%------------------------------------------------
% Research
%------------------------------------------------

% \section{Research}

% \runsubsection{Cornell Robot Learning Lab}
% \descript{| Head Undergrad Research}
%
% \location{Jan 2014 – Present | Ithaca, NY}
% Worked with \textbf{\href{http://www.cs.cornell.edu/~ashesh/}{Ashesh Jain}} and \textbf{\href{http://www.cs.cornell.edu/~asaxena/}{Prof Ashutosh Saxena}} to create \textbf{PlanIt}, a tool which learns from large scale user preference feedback to plan robot trajectories in human environments. Publication submitted.
%
% \sectionspace % Some whitespace after the section

%------------------------------------------------

% \runsubsection{Cornell Phonetics Lab}
% \descript{| Head Undergraduate Researcher}
%
% \location{Mar 2012 – May 2013 | Ithaca, NY}
% Lead the development of \textbf{QuickTongue}, the first ever breakthrough tongue-controlled game with \textbf{\href{http://conf.ling.cornell.edu/~tilsen/}{Prof Sam Tilsen}} to aid in Linguistics research. Publication submitted.
%
% \sectionspace % Some whitespace after the section

%------------------------------------------------
% Awards
%------------------------------------------------

%\section{Awards}
%
%\begin{tabular}{rll}
%2021	 & 1\textsuperscript{st}/20 & Robocode - Free for all \\
%2020	 & 3\textsuperscript{rd}/15 & Robocode - Main tournament \\
%2020     & 3\textsuperscript{rd}/30 & NCPC Qualification (Estonia) \\
%2019     & 1\textsuperscript{st}/25 & Robocode - Main tournament \\
%2018 & 8\textsuperscript{th} & Physics Olympiad Natianal contest \\
%2016 & 1\textsuperscript{st} & Physics Open (Natianal contest) \\
%\end{tabular}

%\sectionspace % Some whitespace after the section

%------------------------------------------------
% Hobbies
%------------------------------------------------

%\section{Hobid}
%
%Purjetamine - Eesti koondise tasemel \\
%Investeerimine, Lugemine\\
%
%\sectionspace % Some whitespace after the section

%----------------------------------------------------------------------------------------

\end{minipage} % The end of the right column

%----------------------------------------------------------------------------------------
%	SECOND PAGE (EXAMPLE)
%----------------------------------------------------------------------------------------

\newpage % Start a new page

\begin{minipage}[t]{0.33\textwidth} % The left column takes up 33% of the text width of the page

\section{Hobid}

Purjetamine


\end{minipage} % The end of the left column
\hfill
\begin{minipage}[t]{0.66\textwidth} % The right column takes up 66% of the text width of the page

%------------------------------------------------
% Projects
%------------------------------------------------

\section{Projektid}

\runsubsection{Elektrirula}
\descript{| Isiklik projekt}

\location{2021 - 2023}
\vspace{\topsep} % Hacky fix for awkward extra vertical space
\begin{tightitemize}
\item Alustasin projektiga, sest tahtsin luua elektrirula, millega saab sõita ilma kiiruse juhtimispulti käes hoidmata.
\item Kiiruse regulleerimise jaoks saadakse andmed \textbf{tensotajuritest} (strain gauge), mis on paigutatud rula väändtelgedele.
\item Mootori kiiruse reguleerimiseks kasutasin \textbf{Arduinot}, mis saab andmed sensoritelt ja väljastab vajaliku PWM signaali.
\end{tightitemize}

\sectionspace % Some whitespace after the section

%------------------------------------------------

\runsubsection{TalTech Student Satellite}
\descript{| Mehaanikainsener}

\location{2022 - 2023}
\begin{tightitemize}
\item Tiimiga ehitasime \textbf{PocketQube} tüüpi satelliiti, mille eesmärgiks oli kosmoses testida uut tüüpi päikesepaneeli ja koguda kuutolmu.
\item Disainisin \textbf{Solidworks} tarkvara abil satelliidile tiibu, millel paiknevad päikesepaneelid.
\item Aitasin kaasa teiste mehaanika- ja tootearendusalaste küsimuste lahendamisel.
\end{tightitemize}

\sectionspace % Some whitespace after the section

\runsubsection{Autonoomne paat}
\descript{| Robotiklubi projekt}

\location{2022}
\begin{tightitemize}
\item Grupitööna disainisime ja ehitasime paadi, mis peab läbima etteantud rada võimalikult kiiresti.
\item Paadi kere ja kõikide osade ühenduslülid on modelleeritud \textbf{Solidworksis} ja \textbf{3D prinditud}.
\item Paadi elektroonika süsteemide kontrollimiseks kasutasime \textbf{STM32 nucleo f303k8}, mis on programmeeritud \textbf{C keeles}.
\item Sensoritena on kasutusel \textbf{IR sensorid}, millega saab mõõta kaugust mingist objektist ja selle abil arvutada paadi optimaalse sõiduteekonna.
\end{tightitemize}

\sectionspace % Some whitespace after the section

\end{minipage} % The end of the right column

%----------------------------------------------------------------------------------------

\end{document}
